% Buch zum Thema "Pendeln mit Rennrad" 
% Begonnen: August 2015
% 

% LH = Latex Hacks
% WAsmL = Wissenschaftliche Arbieten schreiben mit Latex

% Einteilung des Paperformates, des Satzspiegels und eine Bindekorrektur von 1 cm
\documentclass[a4paper,DIV13,BCOR1cm]{scrbook}

\usepackage[english,ngerman]{babel}
\usepackage[T1]{fontenc}
\usepackage[utf8]{inputenc}
\usepackage{lmodern}	% Schrift "Latin Modern" laden
\usepackage{mdwlist}	% Für enger gesetzte Listen

\usepackage{typearea}	% Einstellung des Satzspiegels, siehe Latex Hacks, Hack #31
\usepackage{booktabs} 	% schönere Tabellen, siehe WAsmL, p. 124

\usepackage{url}		% wird von babelbib gebraucht

\usepackage{fancyhdr}
\pagestyle{fancy}

\usepackage{graphicx}

\usepackage{apacite}

\usepackage{eurosym}
%\usepackage{hyperref}

\setcounter{secnumdepth}{3}

\begin{document}

\lhead{Pendeln mit Rennrad}

\title{Pendeln mit dem Rennrad}
\subtitle{Handbuch für die optimale Kombination von Arbeitsweg und Training}  
\author{Marco Strehler}
%\publishers{}
\date{18.\,08.\,2015}
\dedication{Harden The Fuck Up.\\
        Velominati, Rule \#5}
\frontmatter
\maketitle

\chapter{Vorwort}

Das Buch entstand weitgehend auf dem Rad.
D.h. während der Fahrt zur Arbeit und zurück vielen mir die erwähnenswerten Dinge ein, habe sie im Geist schon etwas ausformuliert und zeitnahe dann zu Papier gebracht.

Sirnach, 26.12.2015

\tableofcontents

\mainmatter

\chapter{Einleitung}

\dictum[\protect\citeNP{iglimann2011rennradnews}]{ "Mir geht es oft so, dass ich, bevor ich 'aufsitze', denke,
'...wie bescheuert musst Du eigentlich sein, bei der Dunkelheit und Kälte...',
aber wenn ich dann 5 Minuten gefahren bin, oder auf der Arbeit angekommen bin,
weiß ich warum ich geradelt bin." }

Stimmen von Betroffenen:

\shorthandoff{"}
"Mir geht es oft so, dass ich, bevor ich 'aufsitze', denke,
'...wie bescheuert musst Du eigentlich sein, bei der Dunkelheit und Kälte...',
aber wenn ich dann 5 Minuten gefahren bin, oder auf der Arbeit angekommen bin,
weiß ich warum ich geradelt bin."
\cite{iglimann2011rennradnews}

"Die 2x täglichen 20 Minuten frische Luft sind wunderbar!
Werd ich nass, so what, ich trockne ja auch wieder.
Gibt keine schönere Genugtuung, als an einem kilometerlangen Rattenschwanz an warmgepupsten, im Stau stehenden PKW entlangzudefilieren"
\cite{efix2011rennradnews}


"Ich fühle mich einfach besser mit dem Rad, brauche keinen Parklplatz suchen und bin auf dem Heimweg schneller als mit dem Auto.
UND!!!!! Wenn ich die Firma verlasse ist der Heimweg schon Hobby/Freizeit,
und ich muß nicht ERST mit dem Auto nach Hause."
\cite{littlechex2011rennradnews}

\shorthandon{"}



\chapter{Material}
\section{Das Rennrad}

\section{Regelmässige Wartung des Rades}
Pannensicherheit Rennvelo. Vorsorge.
\begin{figure}[htpb]
        \centering
        \includegraphics[scale=.5]{figs/leistungfahrraeder.eps}
        \caption{Leistungsberechnung}
        \label{fig:leistungfahrraeder}
\end{figure}


\section{Wieso ein Rennrad zum Pendeln}

Es gibt durchaus Gründe, wieso man eher ein Trekking- oder Fitness-Bike als ideales Pendler-Rad sehen kann \cite{Lindthaler2015perfektesfahrradpendler}.



Möglichkeit der Reparatur unterwegs oder am Zielort (Arbeitsplatz, zu Hause). Möglichkeit einer Alternative (ÖV, Taxi).
Reparatur am Arbeitsort
Gelegentlich passiert einem etwas auf dem Weg zur Arbeit. Wenn man das Rennvelo dann reparieren kann, erspart man sich viel umtriebe (Transport des nicht mehr fahrtauglcihen Rades nach Hause, dortige Reparatur). Kleine Reparaturen wie platter Reifen usw. macht man am besten in einer Arbeitsplause oder über Mittag.
Regeneration, Wiederherstellen des Betriebszustand

Das Rennrad muss regelmässig gewartet werden.

Kleine Reperaturen deshalb gleich am Abend machen, am Wochenenden oder freien Tagen einen umfassenden Check (Bremsen, Kette) und Unterhaltsarbeiten.
Je länger man mit Wartungsarbeiten zuwartet, desto umfagreicher werden erfahrungsgemäss die Reparaturen.
Auch eine wichtige Erfahrung: wenn einem beim Fahren etwas auffällt (Geräusch, komisches Gefühl), dann sollte man unbedingt dem nachgehen.
Ignorieren oder Verdrängen (<<Da wird schon nichts sein>>) bringt \emph{todsicher} Probleme.
Irgendwann ist dann die lockere Schraube ganz weg und eine Reparatur, die bei besserer Aufmerksamkeit ein paar Sekunden gedauert hätte,
wird ein Steckenbleiben am blödsten Ort und eine umfangreiche und teure Reparatur.

\chapter{Organisation}

\section{Wie lange ist die Fahrzeit?}

Die Fahrzeit berechnet sich $t = s/v$. Für einen Arbeitsweg von 10\,km und einer Durchschnittsgeschwindigkeit von 30\,km/h kommt man also auf $10/30=0.33$.
Multipliziert mit 60 ergibt die Zeit in Minuten, hier also 20\,min, siehe Tabelle \ref{tab:fahrzeit}.
Die Durchschnittsgeschwindigkeit auf dem Rennvelo ist stark abhängig von der Strecke (flach vs. hüglig) und von den Windverhältnissen.
Kräftiger Gegenwind kann die Durchschnittsgeschwindigkeit gut um 5\,km/h drücken.
Muss man aufgrund der Witterungsverhältnissen einmal auf's MTB umsteigen, dann ist man ebenfalls 5 -- 10\,km/h langsamer.
Die Tabelle \ref{tab:fahrzeit} hilft, sich dabei zu orientieren.

\begin{table}
        \centering
        \begin{tabular}{cccccc}
                \toprule
            &	10\,km	&   20\,km	& 30\,km	&   40\,km	& 50\,km    \\
    \midrule
20\,km/h	&   30      &	60	    & 90        &   120	    & 150       \\
25\,km/h	&   24      &	48 &	72 &	96 &	120  \\
30\,km/h	&   20      &	40 & 	60& 	80 &	100 \\
35\,km/h	&   17      &	34 &	51& 	69 &	86 \\
40\,km/h	&   15      &	30 &	45& 	60 &	75 \\
\bottomrule
        \end{tabular}
        \caption{Fahrzeit in Minuten, abhängig von der Distanz und der Durchschnittsgeschwindigkeit.
        Mit einem Mountainbike ist man zwischen 5 -- 10\,km/h langsamer.
        So kann auch abgeschätzt werden, wieviel länger man für den Arbeitsweg braucht,
        sollten einem einmal die Witterung auf das MTB zwingen.
        Ebenfals rund 5\,km/h langsamer ist man bei starkem Gegenwind (Wetterbericht!).}
        \label{tab:fahrzeit}
\end{table}

\section{Transportierendes Material}

\section{Hygiene}

Ich mache am Arbeitsplatz eine kleine Sport- oder Strandtasche mit allem nötigen parat (Abb. \ref{fig:sporttasche}).
Glücklich, wer eine eigentliche Garderobe mit abschliessbarem Schrank hat.

Aufhängen der Wäsche im Büro.
Um nasse oder verschwitzte Klamotten kommt man nicht herum.
Schön ist, wenn man diese im Büro aufhängen kann.
Allerdings sollte man die Aversion, die solche Klamotten oder nassen Handtücher auslösen können, nicht unterschätzen.
Was für einem selber die ultimative Trophäe und Beweis seiner Leistungsfähigkeit ist, ist für andere nur eine blanke Zumutung.

\begin{figure}[htpb]
        \centering
        \includegraphics[width=\textwidth]{figs/sporttasche.eps}
        \caption{Im Büro ist eine immer gepackte Sporttasche, in der die Wechselkleider und das Duschzeugs ist.}
        \label{fig:sporttasche}
\end{figure}
\chapter{Pendeln und Training}

\shorthandoff{"}
    \dictum[\citeNP{sollritchey2011rennradnews}]{
        "Ich fahre jeden Tag 33 km einfach zur Arbeit und das bei JEDEM Wetter.
        [\ldots]. Im Winter gibt es kein besseres Training,
        ohne Arbeit würde ich diese km Leistung niemals fahren"
        }
\shorthandon{"}

\section{Problematik}

Bei 100prozentigem Arbeitspensum und allenfalls Familie noch genügend Zeit
für ein seriöses Rennrad-Training aufzubringen ist eine Herausforderung.
Eine mögliche Lösung ist das regelmässige Pendeln mit dem Rennrad (road
bike commuting). In diesem Artikel werden die Schwierigkeiten und mögliche
Lösungen zum Pendeln mit Rennrad (PmRR) dargestellt. Der Autor kann dabei
auf mehrere Jahre Erfahrung mit einem Wochenschnitt von ca. 100 km pendelnd
zurückschauen.

Pendeln zum Arbeitsplatz und Rennrad-Training sind zwei unterschiedliche
Tätigkeiten, die sich nur in einem kleinen Bereich überschneiden, nämlich
in der Eigenschaft, dass man sich (möglichst schnell) vom Punkt A nach Punkt
B bewegt.

In vielen anderen Bereichen sind hier konträre Ziele: beim Pendeln will man
möglichst grosse Flexibilität, gepaart mit möglichst viel Bequemlichkeit.
Man muss oft Dinge (Unterlagen, Bücher, Equipment) transportieren. Der
Arbetisplatz stellt Anforderung an Erscheinungsbild (Hygiene, Kleidung). Der
Arbeitsplatz muss pünktlich erreicht werden -- dies auch bei schlechtem
Wetter oder bei Dunkelheit. Auch will man pünktlich wieder zu Hause sein.

Beim Rennrad-Training will man möglichst \emph{wenig} mitnehmen auf einem
Rad, dass möglichst leicht ist. Die Kleidung soll für das Rannradfahren sehr
funktional sein. Um den Trainingeffekt zu optimieren ist es unvermeidlich,
dass man Schwitzt. Rennradfahren bei Dunkelheit, viel Verkehr oder schlechten
Wetter wird -- wenn möglich -- vermieden.

Vorteile:
\begin{itemize}
        \item Zeit, die man für den Arbeitsweg aufbringt wird als Trainingszeit genützt.
        \item Das Training wird am Tag gesplittet.
        \item Allenfalls Kostenersparnis gegenüber der Benutzung von anderen privaten Verkehrsmitteln (Auto) oder öffentlichem Verkehr.
        \item Keine Abhängigkeit von öffentlichem Verkehr.
\end{itemize}
\section{Faktoren des Trainings}

Häufigkeit, Dauer, Intensität

\section{Häufigkeit}

\section{Dauer}

\begin{table}
        \centering
        \begin{tabular}{ccccccccccc}
                \toprule
&	5	& 10	& 15	& 20	& 25	& 30 & 35	& 40	& 45	& 50\\
    \midrule
1 &	480	& 960	& 1440	& 1920	& 2400	& 2880	& 3360	& 3840	& 4320	& 4800 \\
2 &	960 &	1920 &	2880 &	3840 &	4800 &	5760 &	6720 &	7680 &	8640 &	9600 \\
3 &	1440 & 	2880 &	4320 &	5760 &	7200 &	8640 &	10080 &	11520 &	12960 &	14400 \\
4 &	1920 &	3840 &	5760 &	7680 &	9600 &	11520 &	13440 &	15360 &	17280 &	19200 \\
5 &	2400& 	4800 &	7200 &	9600 &	12000 &	14400 &	16800 &	19200 &	21600 &	24000 \\
\bottomrule
        \end{tabular}
        \caption{Jahreskilometer abhängig von der Strecke Wohnort-Arbeitsort sowie der Anzahl Tage, an denen mit dem Rad gependelt wird.}
        \label{tab:jahreskilometer}
\end{table}

\section{Intensität}



\chapter{Sicherheit}

\section{Diebstahlschutz}

Etwas vom Allerärgerlichsten ist wohl der Diebstahl des Rades. Geschichen,
von Personen, die sich <<nur kurz umgedreht haben>>, und das neue, teure
Carbon-Rad war weg gibt es genug. Die Situation wird für Pendler, die das
Rad an einem öffentlichen Ort oder auf dem Firmen-Rad-Unterstand für die
Arbeitszeit anbringen müssen, nicht wirklich besser

\begin{enumerate}
  \item Das Rad immer an einem festen Gegenstand sichern. Das Abschliessen von Vorder- oder Hinterrad ist bloss eine 
    Wegfahr, aber keine Wegtragsperre.

  \item Beim Abschliessen darauf achten, dass das Verschlusssystem (Kette, Bügelschloss) eng sitzt.
    Dies um einem Dieb möglichst wenig Arbeitsraum zu bieten. Das Schloss soll möglichst nach unten zeigen und
    schwer zugänglich sein.

  \item Das Rad für den Arbeitsweg sollte eher vom Typ <<Stadtschlampe>>, als dem ultimativen Renner sein.
    Siehe dazu auch das Kapitel <<Welches Rad?>>

  \item Mit zwei Rädern und zwei Schlössern kann man Schliessgemeinschaften bilden.
    Ein Dieb muss so zwei Schlösser knacken, um sich mit der Beute aus dem Staub machen zu können.

  \item Entgegen dem Impus, das Rad etwas <<verstecken>> zu wollen, wähle eine belebte Stelle um das Rad zu sichern.
    Auf grossen Rad-Abstellplätzen eine vordere Reihe wählen.
    Im Blick der Passanten ist das Rad besser geschützt.
    Mir ist es allerdings schon passiert, dass ich das eigene Rad wg. eines Schlossdefektes selber knacken musste.
    Obwohl ich dabei mit einem grossen Bolzenschneider in aller Öffentlichkeit vorging, wurde ich nicht aufgehalten.
    Der oft geäusserte Tipp scheint also nur sehr beschränkt zu wirken.

  \item Ersatz von Schnellspannern an Laufrad oder Sattel mit herkömmlicher Schraube ersetzen.
    Es gibt auch sog. Tamper-proofe Schrauben -- für den Sattel. Allerdings muss man dann auch die entsprechenden WErkzeuge
    für eine Panne mitführen. Allenfalls hat dann bei einer Panne auch ein Bike-Werkstatt das entsprechende Werkeug nicht.

  \item Was sind die sichersten Rad-Schlösser? Mögliche Kaufempfehlung bietet http://www.trelock.de/

\end{enumerate}

\section{Einhalten von Verkehrsregeln}
Ein wichtiger Punkt zur Risikosenkung scheint mir das Einhalten der Verkehrsregeln. Der Arbeitsweg ist i.d.R. nicht ein idealer Velo-Weg. Oft ist der Verkehr nicht entflochten und das Verkehrsaufkommen ist gross. Zudem sind Autopendler oft gestresst, abgelenkt, müde.  Das peinliche Einhalten von Verkehrsregeln scheint mir aus folgenden Gründen angepasst:
Das Risiko wird erheblich gemindert. Man wird für die anderen Verkehrsteilnehmer berechenbarer.
Man provoziert keine anderen Verkehrsteilnehmer.
Sollte tatsächlich etwas passieren, ist man rechtlich auf der sicheren Seite.
\cite{Flieshardt2015}

\subsection{Strassenverkehrsregeln Radfahrer Deutschland}
\citeNP{Flieshardt2015}
\begin{itemize}
        \item Radwegbenutzung: Bezeichnete Radwege (Zeichen 237, 240, 241) müssen benutzt werden. Ansonsten ist die Nutzung von Radwegen freiwillig.
        \item Nebeneinanderfahren: Solange der Verkehr nicht behindert wird. 
        \item Einbahnstrasse in Gegenrichtung: Erlaubt mit Zusatzschild <<Radfahrer frei>>.
        \item Musik hören: Warnsignale müssen wahrgenommen werden.
        \item Beleuchtung: Front- und Rücklicht müssen jederzeit mitgeführt werden. Akkulampen benötigen ein StVZO-Siegel.
\end{itemize}

\subsection{Strassenverkehrsregeln Radfahrer Oesterreich}
\begin{itemize}
        \item Radwegbenutzung: Radwege müssen benutzt werden. Trainierede Rennradfahrer sind von der Pflicht ausgenommen.
        \item Nebeneinanderfahren: im Training erlaubt. 
        \item Einbahnstrasse in Gegenrichtung: Gestattet, wenn Erlaubnis gesondert beschildert.
        \item Musik hören: nicht geregelt, kein ausdrückliches Verbot
        \item Beleuchtung: müssen bei guter Sicht nicht mitgeführt werden.
\end{itemize}

\subsection{Strassenverkehrsregeln Radfahrer Schweiz}
\begin{itemize}
        \item Radwegbenutzung: müssen auch von Rennradfahrern benutzt werden
        \item Nebeneinanderfahren: bei geringem Verkehr erlaubt.
        \item Einbahnstrasse in Gegenrichtung: Erlaubt wenn gesondert ausgeschildert.
        \item Musik hören: Erlaubt, wenn Warnsignale wahrnehmbar sind.
        \item Beleuchtung: müssen bei guter Sicht nicht mitgeführt werden.
\end{itemize}

\section{Helm}

\section{Licht/Reflektoren}

Genügend Licht ist unverzichtbar. Es gibt nur wenige Sommermonate, wo man mit Sicherheit bei vollem Tageslicht zur und von der Arbeit kommt.
Meist wird es schon im September morgens schon so dunkel, dass es die Sicherheit gebietet, sich mit Licht zu behängen.

Optimalerweise nimmt man eine Lampe am Lenker und am Helm. Licht am Helm hat den vorteil, dass man bei einer Kopfbewegung
noch weitere Teile ausleuchten kann (Kettenposition!). Man wird aber auch wg. der erhöhten Pos. des Lichtes
von Verkehrsteilnehmer besser wahrgenommen.

\chapter{Arbeitplatzaspekte}

\section{Positive Aspekte für die Arbeit}

\section{Negative Aspekte}

\subsection{Mögliche Konflikte am Arbeitsplatz}

Wer wöchentlich 100km oder mehr auf dem Rad zurücklegt ist in Kreisen von Strava und Fitocracy allenfalls Durchschnitt.
Vorsichtig mit Angeben der sportlichen Leistungen vor Kollegen, insbesondere auch Untergebenen und Vorgesetzten.
Es kann sein, dass hier auch Neid mitspielt, aber wer zu sehr auf seine sportlichen Erfolge pocht,
der stösst in durchschnittlich sportlichen Kreisen durchaus auch auf Ablehnung.
Inbesondere wenn Absenzen durch Krankheit, Unfall oder Wettkämpfe dazukommen:
die berufliche Leistung darf auf gar keinen Fall darunter leiden, sonst kann man sicher sein, dass man schnell als "Verrückter" oder "Narzisst" gilt.
Etwas "low profile" kann nicht schaden. Ich würde insbesondere hier auch auf eine strikte Trennung von privat und geschäft Raten: Stöbern in Rennrad-Foren, Unterhalten eines (Sport-)-Blotg, Nachfüren von Webeinträgen in Strava und Fitocracy wären totale No-Gos am Arbeitsplatz.

http://www.commutebybike.com

Wenn man jeden Tag eine sportliche Leistung erbringen muss, stellt sich die Frage nach einer entsprechenden Regeneration. Wenn die Regeneration ausbleibt, fehlt irgendwann die Leistung.
Nahrungs- und Flüssigkeitsaufnahme: Auf eine entsprechende Menge an Kohlenhydraten und Proteinen (Muskelreparatur) muss geachtet werden. Den Flüssigkeitsverlust muss ausgeglichen werden.
Alkohol verzögert die Regeneration. Man sollte deshalb darauf verzichten.



\chapter{Schlechtes Wetter}

\shorthandoff{"}
\dictum[\citeNP{nordwind2012rennradnews}]{"Direkt nach den Aufstehen aus dem Fenster geguckt und es war am Regnen und sehr Windig,
20min später losgefahren und siehe da der Regen hatte auf gehört und ich hatte den kompleten Weg zur Arbeit Rückenwind wie doof."}

\shorthandon{"}

Wenn man sich entscheidet, mit dem Rennrad zu pendeln gehört eine positive Grundhaltung zu \emph{jedem} Wetter dazu.
Es braucht dazu auch ein gewisses Mass an kognitiver Umstrukturierung. Bei schönem Wetter fahren kann jeder.
Es ist durchaus so, dass ich bei mildem, sonnigen Wetter (nicht zu heiss, nicht zu kalt) am liebsten fahre.
Nun ist das halt nicht immer der Fall. Die Schweiz hat gleichmässig etwa 12 bis 14 Regentage pro Monat.
D.h. dass im Schnitt es so an jedem dritten Tag mit Regen zu rechnen ist.
Wenn man Glück hat, sitzt man nicht gerade im Sattel, wenn dieses Nass vom Himmel kommt, sondern vorher oder nachher.
Trotzdem ist mit Nasswerden auch bei optimaler Planung und Studium des Wetterbrichtes immer zu rechnen.

Ein weiterer Trick ist, sich selbst kognitiv neu zu strukturieren (sprich: die Sache positiv zu sehen).
Zum Beispiel sich bewusst zu machen, dass die relativ kurze Distanz zur Arbeit nicht ausreicht, 
um wirklich auszukühlen oder wie Bradley Wiggins meint: <<It's not really long enough to get super-cold>> \cite{bbc2015wigginswinter}.
Für weitere Beispiele für die persönliche kognitive Umstrukturierung siehe Tabelle \ref{tab:kognitiveumstrukturierung}).

\begin{table}
        \centering
        \begin{tabular}{l}
                \toprule
        <<Nach 5 Minuten im Sattel spüre ich die Kälte nicht mehr.>>\\
        <<Genau jetzt hole ich mir den Trainingsvorteil gegenüber Schönwetterfahrern.>>\\
        <<Jetzt verbessere ich meine Regenfahrtechnik.>>\\
        <<Ich hole mir jetzt eine monstermässige Rennhärte!>>\\
        <<Bei schönem Wetter fahren kann jedes Weichei.>>\\
        <<Nichts ist schöner, als nach einer Regenfahrt unter die warme Dusche zu stehen.>>\\
        <<Harden The Fuck Up!>> \cite[Rule \#5]{velominati2014rules}\\
        % <<If you are out riding in bad weather, it means you are a badass. Period. \cite[Rule \#9]{velominati2014rules}\\
                \bottomrule
        \end{tabular}
        \caption{Kognitive Umstrukturierung: wichtig ist dabei sich einen für sich stimmige Grundsatzüberzeugung zu finden.
        Diese muss dann möglichst oft ins Bewusstsein geholt werden, um verankert zu werden.}
        \label{tab:kognitiveumstrukturierung}
\end{table}

\chapter{Probleme des realen Lebens}

\section{Weitere Informationen}

Die Fahrstrecke kann genutzt werden, um Techniken (Haltung, Trittfrequenz, Wiegeschritt) zu üben.

Täglich an den gleichen Ort zu fahren, hat den Vorteil, dass man Feintuning betreiben kann.
Es empfiehlt sich, gefährliche Kreuzungen oder Strecken allenfalls zu umfahren oder eine Alternative zu suchen.
Auch kann mit der Abfahrtszeit gespielt werden. Schon 10 Minuten früher oder später kann bewirken, dass der Verkehr deutlich weniger ist oder weniger Lastwagen unterwegs sind.

Schlaglöcher sind mit der Zeit sehr vertraut und man kann schon frühzeitig eine andere Linie einschlagen.


\section{Fahrradtransport mit öffentlichen Verkehrsmitteln}

Die hier zusammengetragenen Informationen beziehen sich auf den Nahverkehr.
Oft sind abweichende Bestimmung im Bezug auf die Region oder des lokalen Verkehrsverbundes vorhanden.
Ein Problem ist auch, dass die einfache Fahrradmitnahme gerade zur Pendlerzeit oft nicht erlaubt oder
von (wenig planbarer) Einschätzung des Zugspersonal abhängig ist.
Insgesamt unproblematischer scheint es zu sein, das Fahrrad teildemontiert in eine entsprechende Tasche zu packen und als Handgepäck mitzuführen.

\subsection{Deutsche Bahn (http://www.bahn.de)}
Im Nahverkehr mit Fahrradsymbol kann das Fahrrad mitgenommen werden.
Zeiten mit Berufsverkehr sollen jedoch gemieden werden.
Eine ausdrückliches Mitnahmerecht besteht nicht und das letzte Wort hat das Zugspersonal. Zudem gibt es unterschiedliche Regelungen je nach Region.

Es gibt im DB Nahverkehr eine Fahrradtageskarte für \EUR{5}. Innerhalb von Verkehrsverbünden existieren teilweise abweichende Tarifbestimungen.

Als Handgepäck: Auf der Website der Bahn finden sich keine ausdrücklichen Angaben zur Mitnahme des verpackten, teilzerlegten Fahrrades.
Erfahrungsberichte bestätigen aber die Möglichkeit der Mitnahme.

\subsection{Österreichische Bundesbahn (http://www.oebb.at)}
Im Nahverkehr kann die Fahrradmitnahme nur bei genügend freien Stellplätzen erfolgen.
Ein Fahrradticket kostet 10\% des Vollpreises der 2. Klasse, mindestens aber \EUR{2}.
Es gibt zudem Wochen- und Monatskarten.

Die Mitnahme von teildemontierten und verpackten Fahrrädern ist in allen Zügen kostenfrei möglich.

\subsection{Schweizerische Bundesbahnen (http://www.sbb.ch)}
Eine Fahrradmitnahme ist im Nahverkehr (S-Bahnen) möglich, allerdings von Montag bis Freitag nur von 8 bis 16 Uhr und von 19 bis 6 Uhr.
Ein Fahrrad braucht grundsätzlich das gleiche Ticket wie der Passagier, d.h. den vollen Preis ohne Halbtax-Abo.
Günstiger geht es mit dem Velo-Billet, das es als Velo-Tageskarte (CHF\,18) oder sog. Velo-Pass (1 Jahr, CHF\,220) gibt.

Wenn Sie Ihr Velo in einer Tragetasche verpacken, können Sie es kostenlos als Handgepäck im Zug mitnehmen.
Jede Hülle wird akzeptiert. Das Vorderrad muss demontiert und zusammen mit dem Velo in der Transporthülle verpackt sein.
Verstauen Sie Ihr verpacktes Velo während der Zugfahrt unter bzw. über dem Sitz oder im Einsteigebereich.
Wenn Sie Ihre Velotragetasche auf einem Sitzplatz deponieren möchten, müssen Sie zusätzlich ein halbes Billett lösen.


\section{Fahren trotz körperlichen Symptomen}

\subsection{Zeichen von Uebertraining}

\subsection{Schlafmangel}

\subsection{Fahren mit einer Infektion}

\subsection{Fahren mit Restalkohol}

\subsection{Radfahren mit Kater}
\cite{Hurford2015hangover}

Mögliche Faktoren für Kater-Symptome \cite{swift1998alcohol} sind direkte Effekte des Alkohols.
Dehydratation, Elektrolytengleisung, Gastrointestinale Probleme, tiefer Blutzucker, Schlafstörung
Indirekte Wirkungen des Alkoholkonsumes:
Alkoholenzugssymptome, Abbauprodukte des Alkoholabbaus (Acetaldehyd)
Zusätzlich wirken noch nicht alkoholbezogene Faktoren: Methanol, Fuselalkohole, Nikotin

Flüssigkeit mit Elektrolyten (Konkret?)

Gegen Kopf- und Gleiderschmerzen helfen konventionelle, rezeptfrei erhältliche Schmerzmittel.
Speziell geeignet bei Kater-Kopfschmerzen ist \textsl{Ibuprofen}
(enthalten beispielsweise in Brufen, Contra Schmerz und Dolocyl).
\textsl{Acetylsalicylsäure} (Aspirin) geht auch, kann aber zusätzlich auf den Magen schlagen.
Und wenn wir gerade dabei sind: \textsl{Ibuprofen} wie auch \textsl{Acetylsalicylsäure}
sind Medikamente, die nicht unter die Kategorie Doping fallen \cite{nada2014erlaubtemedikamente}.
Das wären also Medikamente, die in einer solchen Situation vor einem Wettkampf genommen werden könnten.
Abgeraten wird von Paracetamol (auch in vielen Grippemitteln enthalten),
weil \textsl{Paracetamol} den gleichen Abbauweg wie Alkohol nimmt und so zusätzlich die Leber belastet.

Kohlenhydrate (Blutzuckersenkung durch Alkohol)

Niedrige Belastung, hohe Kadenz (Laktat-Abbau)

Vermeidung Intervalle wg. Kopfschmerzen und Bauchbeschwerden

\backmatter

\bibliographystyle{apacite}
\bibliography{pendeln.bib}

\end{document}
