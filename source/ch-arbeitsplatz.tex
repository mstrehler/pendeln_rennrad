\chapter{Arbeitplatzaspekte}

\section{Positive Aspekte für die Arbeit}

\section{Negative Aspekte}

\subsection{Nicht mit sportlichen Erfolgen angeben}

Wer wöchentlich 100\,km oder mehr auf dem Rad zurücklegt ist in Kreisen von Fitness-Websites wie Strava und Fitocracy allenfalls Durchschnitt.
Hier ist Raum um sich mit den sportlichen Leistungen zu brüsten.
Vorsicht ist allerdings Geboten herausposaunen der sportlichen Leistungen vor Arbeitskollegen, insbesondere auch Untergebenen und Vorgesetzten.
Es kann sein, dass hier auch Neid mitspielt, aber wer zu sehr auf seine sportlichen Erfolge pocht,
der stösst in durchschnittlich sportlichen Kreisen der Couch-Potatos durchaus auf Ablehnung.
Inbesondere wenn Absenzen durch Krankheit, Unfall oder Wettkämpfe dazukommen.
Die berufliche Leistung darf auf gar keinen Fall darunter leiden, sonst kann man sicher sein, dass man schnell als "Verrückter" oder "Narzisst" gilt.

Ohne Optimierung der Abläufe lässt sich ein Vollzeitjob und intensives Training nicht verbinden \cite{Roemer2014ironmanvollzeitjob}.
In diesem Artikel von Römer wird auch Michael Krell zitiert: <<Hobbysportler [sollten] im Beruf eher sparsam mit Geschichten zu ihrem Trainingseifer umgehen>>,
<<Auf keinen Fall darf bei Chefs der Eindruck entstehen, dass der Sport die Arbeit negativ beeinflusst ...>>.

Etwas \emph{low profile} kann also nicht schaden.

\subsection{Vorsicht mit sozialen Medien und Surfen}

Ich würde insbesondere hier auch auf eine strikte Trennung von Privat und Geschäft raten.
Stöbern in Rennrad-Foren, Unterhalten eines (Sport-)-Blotg, Nachfüren von Webeinträgen in Strava und Fitocracy sind meines Erachtens totale No-Gos am Arbeitsplatz.


