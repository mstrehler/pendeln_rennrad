\chapter{Arbeitplatzaspekte}

\section{Positive Aspekte für die Arbeit}

\section{Negative Aspekte}

\subsection{Mögliche Konflikte am Arbeitsplatz}

Wer wöchentlich 100km oder mehr auf dem Rad zurücklegt ist in Kreisen von Strava und Fitocracy allenfalls Durchschnitt.
Vorsichtig mit Angeben der sportlichen Leistungen vor Kollegen, insbesondere auch Untergebenen und Vorgesetzten.
Es kann sein, dass hier auch Neid mitspielt, aber wer zu sehr auf seine sportlichen Erfolge pocht,
der stösst in durchschnittlich sportlichen Kreisen durchaus auch auf Ablehnung.
Inbesondere wenn Absenzen durch Krankheit, Unfall oder Wettkämpfe dazukommen:
die berufliche Leistung darf auf gar keinen Fall darunter leiden, sonst kann man sicher sein, dass man schnell als "Verrückter" oder "Narzisst" gilt.
Etwas "low profile" kann nicht schaden. Ich würde insbesondere hier auch auf eine strikte Trennung von privat und geschäft Raten: Stöbern in Rennrad-Foren, Unterhalten eines (Sport-)-Blotg, Nachfüren von Webeinträgen in Strava und Fitocracy wären totale No-Gos am Arbeitsplatz.

http://www.commutebybike.com

Wenn man jeden Tag eine sportliche Leistung erbringen muss, stellt sich die Frage nach einer entsprechenden Regeneration. Wenn die Regeneration ausbleibt, fehlt irgendwann die Leistung.
Nahrungs- und Flüssigkeitsaufnahme: Auf eine entsprechende Menge an Kohlenhydraten und Proteinen (Muskelreparatur) muss geachtet werden. Den Flüssigkeitsverlust muss ausgeglichen werden.
Alkohol verzögert die Regeneration. Man sollte deshalb darauf verzichten.



