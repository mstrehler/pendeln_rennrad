\chapter{Schlechtes Wetter}

\shorthandoff{"}
\dictum[\citeNP{nordwind2012rennradnews}]{"Direkt nach den Aufstehen aus dem Fenster geguckt und es war am Regnen und sehr Windig,
20min später losgefahren und siehe da der Regen hatte auf gehört und ich hatte den kompleten Weg zur Arbeit Rückenwind wie doof."}

\shorthandon{"}

Wenn man sich entscheidet, mit dem Rennrad zu pendeln gehört eine positive Grundhaltung zu \emph{jedem} Wetter dazu.
Es braucht dazu auch ein gewisses Mass an kognitiver Umstrukturierung. Bei schönem Wetter fahren kann jeder.
Es ist durchaus so, dass ich bei mildem, sonnigen Wetter (nicht zu heiss, nicht zu kalt) am liebsten fahre.
Nun ist das halt nicht immer der Fall. Die Schweiz hat gleichmässig etwa 12 bis 14 Regentage pro Monat.
D.h. dass im Schnitt es so an jedem dritten Tag mit Regen zu rechnen ist.
Wenn man Glück hat, sitzt man nicht gerade im Sattel, wenn dieses Nass vom Himmel kommt, sondern vorher oder nachher.
Trotzdem ist mit Nasswerden auch bei optimaler Planung und Studium des Wetterbrichtes immer zu rechnen.

Ein weiterer Trick ist, sich selbst kognitiv neu zu strukturieren (sprich: die Sache positiv zu sehen).
Zum Beispiel sich bewusst zu machen, dass die relativ kurze Distanz zur Arbeit nicht ausreicht, 
um wirklich auszukühlen oder wie Bradley Wiggins meint: <<It's not really long enough to get super-cold>> \cite{bbc2015wigginswinter}.
Für weitere Beispiele für die persönliche kognitive Umstrukturierung siehe Tabelle \ref{tab:kognitiveumstrukturierung}).

\begin{table}
        \centering
        \begin{tabular}{l}
                \toprule
        <<Nach 5 Minuten im Sattel spüre ich die Kälte nicht mehr.>>\\
        <<Genau jetzt hole ich mir den Trainingsvorteil gegenüber Schönwetterfahrern.>>\\
        <<Jetzt verbessere ich meine Regenfahrtechnik.>>\\
        <<Ich hole mir jetzt eine monstermässige Rennhärte!>>\\
        <<Bei schönem Wetter fahren kann jedes Weichei.>>\\
        <<Nichts ist schöner, als nach einer Regenfahrt unter die warme Dusche zu stehen.>>\\
        <<Harden The Fuck Up!>> \cite[Rule \#5]{velominati2014rules}\\
        % <<If you are out riding in bad weather, it means you are a badass. Period. \cite[Rule \#9]{velominati2014rules}\\
                \bottomrule
        \end{tabular}
        \caption{Kognitive Umstrukturierung: wichtig ist dabei sich einen für sich stimmige Grundsatzüberzeugung zu finden.
        Diese muss dann möglichst oft ins Bewusstsein geholt werden, um verankert zu werden.}
        \label{tab:kognitiveumstrukturierung}
\end{table}

