\chapter{Sicherheit}

\section{Diebstahlschutz}

Etwas vom Allerärgerlichsten ist wohl der Diebstahl des Rades. Geschichen,
von Personen, die sich <<nur kurz umgedreht haben>>, und das neue, teure
Carbon-Rad war weg gibt es genug. Die Situation wird für Pendler, die das
Rad an einem öffentlichen Ort oder auf dem Firmen-Rad-Unterstand für die
Arbeitszeit anbringen müssen, nicht wirklich besser

\begin{enumerate}
  \item Das Rad immer an einem festen Gegenstand sichern. Das Abschliessen von Vorder- oder Hinterrad ist bloss eine 
    Wegfahr, aber keine Wegtragsperre.

  \item Beim Abschliessen darauf achten, dass das Verschlusssystem (Kette, Bügelschloss) eng sitzt.
    Dies um einem Dieb möglichst wenig Arbeitsraum zu bieten. Das Schloss soll möglichst nach unten zeigen und
    schwer zugänglich sein.

  \item Das Rad für den Arbeitsweg sollte eher vom Typ <<Stadtschlampe>>, als dem ultimativen Renner sein.
    Siehe dazu auch das Kapitel <<Welches Rad?>>

  \item Mit zwei Rädern und zwei Schlössern kann man Schliessgemeinschaften bilden.
    Ein Dieb muss so zwei Schlösser knacken, um sich mit der Beute aus dem Staub machen zu können.

  \item Entgegen dem Impus, das Rad etwas <<verstecken>> zu wollen, wähle eine belebte Stelle um das Rad zu sichern.
    Auf grossen Rad-Abstellplätzen eine vordere Reihe wählen.
    Im Blick der Passanten ist das Rad besser geschützt.
    Mir ist es allerdings schon passiert, dass ich das eigene Rad wg. eines Schlossdefektes selber knacken musste.
    Obwohl ich dabei mit einem grossen Bolzenschneider in aller Öffentlichkeit vorging, wurde ich nicht aufgehalten.
    Der oft geäusserte Tipp scheint also nur sehr beschränkt zu wirken.

  \item Ersatz von Schnellspannern an Laufrad oder Sattel mit herkömmlicher Schraube ersetzen.
    Es gibt auch sog. Tamper-proofe Schrauben -- für den Sattel. Allerdings muss man dann auch die entsprechenden WErkzeuge
    für eine Panne mitführen. Allenfalls hat dann bei einer Panne auch ein Bike-Werkstatt das entsprechende Werkeug nicht.

  \item Was sind die sichersten Rad-Schlösser? Mögliche Kaufempfehlung bietet http://www.trelock.de/

\end{enumerate}

\section{Einhalten von Verkehrsregeln}
Ein wichtiger Punkt zur Risikosenkung scheint mir das Einhalten der Verkehrsregeln. Der Arbeitsweg ist i.d.R. nicht ein idealer Velo-Weg. Oft ist der Verkehr nicht entflochten und das Verkehrsaufkommen ist gross. Zudem sind Autopendler oft gestresst, abgelenkt, müde.  Das peinliche Einhalten von Verkehrsregeln scheint mir aus folgenden Gründen angepasst:
Das Risiko wird erheblich gemindert. Man wird für die anderen Verkehrsteilnehmer berechenbarer.
Man provoziert keine anderen Verkehrsteilnehmer.
Sollte tatsächlich etwas passieren, ist man rechtlich auf der sicheren Seite.
\cite{Flieshardt2015}

\subsection{Strassenverkehrsregeln Radfahrer Deutschland}
\citeNP{Flieshardt2015}
\begin{itemize}
        \item Radwegbenutzung: Bezeichnete Radwege (Zeichen 237, 240, 241) müssen benutzt werden. Ansonsten ist die Nutzung von Radwegen freiwillig.
        \item Nebeneinanderfahren: Solange der Verkehr nicht behindert wird. 
        \item Einbahnstrasse in Gegenrichtung: Erlaubt mit Zusatzschild <<Radfahrer frei>>.
        \item Musik hören: Warnsignale müssen wahrgenommen werden.
        \item Beleuchtung: Front- und Rücklicht müssen jederzeit mitgeführt werden. Akkulampen benötigen ein StVZO-Siegel.
\end{itemize}

\subsection{Strassenverkehrsregeln Radfahrer Oesterreich}
\begin{itemize}
        \item Radwegbenutzung: Radwege müssen benutzt werden. Trainierede Rennradfahrer sind von der Pflicht ausgenommen.
        \item Nebeneinanderfahren: im Training erlaubt. 
        \item Einbahnstrasse in Gegenrichtung: Gestattet, wenn Erlaubnis gesondert beschildert.
        \item Musik hören: nicht geregelt, kein ausdrückliches Verbot
        \item Beleuchtung: müssen bei guter Sicht nicht mitgeführt werden.
\end{itemize}

\subsection{Strassenverkehrsregeln Radfahrer Schweiz}
\begin{itemize}
        \item Radwegbenutzung: müssen auch von Rennradfahrern benutzt werden
        \item Nebeneinanderfahren: bei geringem Verkehr erlaubt.
        \item Einbahnstrasse in Gegenrichtung: Erlaubt wenn gesondert ausgeschildert.
        \item Musik hören: Erlaubt, wenn Warnsignale wahrnehmbar sind.
        \item Beleuchtung: müssen bei guter Sicht nicht mitgeführt werden.
\end{itemize}

\section{Helm}

\section{Licht/Reflektoren}

Genügend Licht ist unverzichtbar. Es gibt nur wenige Sommermonate, wo man mit Sicherheit bei vollem Tageslicht zur und von der Arbeit kommt.
Meist wird es schon im September morgens schon so dunkel, dass es die Sicherheit gebietet, sich mit Licht zu behängen.

Optimalerweise nimmt man eine Lampe am Lenker und am Helm. Licht am Helm hat den vorteil, dass man bei einer Kopfbewegung
noch weitere Teile ausleuchten kann (Kettenposition!). Man wird aber auch wg. der erhöhten Pos. des Lichtes
von Verkehrsteilnehmer besser wahrgenommen.

