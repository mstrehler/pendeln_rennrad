\chapter{Einleitung}

\dictum[\protect\citeNP{iglimann2011rennradnews}]{ "Mir geht es oft so, dass ich, bevor ich 'aufsitze', denke,
'...wie bescheuert musst Du eigentlich sein, bei der Dunkelheit und Kälte...',
aber wenn ich dann 5 Minuten gefahren bin, oder auf der Arbeit angekommen bin,
weiß ich warum ich geradelt bin." }

\section{Training, Arbeit, Pendeln}

Das Buch fasst meine Erfahrungen mit Rennrad-Pendeln zusammen.
Es ist insbesondere der Versuch \emph{Training}, \emph{Arbeit} und \emph{Pendeln} möglichst optimal zu verbinden.
Die Schwierigkeit liegt meines Erachtens darin, dass die einzelnen Bereiche für sich gesehen relativ starr erscheinen.

Das Training (der Begriff steht hier in Abgrenzung zum unstrukturierten blossen "Fitness betreiben") folgt einem Programm
und ist zielgerichtet (z.B. hin auf einen Wettkampf oder der Verbesserung bestimmter Fähigkeiten).
Ein optimales Training konkurriert durch den zeitlichen Umfang, einer notwendigen Periodisierung sowie
begleitenden Massnahmen (Ernährung, Unterhalt Material) zwangszweise mit anderen Lebensbereichen.

Die Arbeit ist an Zeit und Ort gebunden und erfordert ein hohes Mass der persönlichen Leistungsfähigkeit.
Die Arbeit ist zudem eine Tätigkeit, für die man hochmotiviert ist.
In der Regel ist es wohl auch eine externe Motivation (Geld, Angst vor Arbeitslosigkeit),
das Spielt aber an dieser Stelle keine Rolle.

Das Pendeln bedingt das physische Bewegen des eigenen Körpers zwischen dem Wohn- und Arbeitsortes.
Dies soll zudem zuverlässig, pünktlich und sicher erfolgen.
Das Pendeln soll dabei auch ökonomisch Sinn machen (kurze Rüstzeit, tiefe Kosten).

Stellt man sich die drei Bereiche als Eckpunkte eines Dreiecks vor, ergeben sich drei Schnittstellen,
die betrachtet und optimiert werden müssen.
Wie wir sehen werden, ergeben sich bei den Schnittpunkten nicht nur zu lösende Herausforderungen,
sondern auch eine grosse Chance für Synergien und wechselseite Verstärkung.
Um eine optimale Verbindung der Bereiche zu erreichen müssen also die Herausforderungen einzeln maximal gelöst
und die Chancen maximal verstärkt werden.

\subsection{Schnittstelle Training--Arbeit}

\minisec{Herausforderungen}

  \begin{itemize*}
    \item Zeit und Energie, die ich in die Arbeit investiere, fehlt mir im Training. Und umgekehrt.
    \item Sportliches Engagement kann vom Arbeitgeber oder Mitarbeitern kritisch gesehen werden.
      Sei es durch Neid der Unfitten oder durch die Wahrnehmung, dass der Fokus zu stark beim Training ist.
    \item Schwierigkeiten der Umsetzung eines Trainingsplanes durch unregelmässige Arbeitszeiten.
  \end{itemize*}

\minisec{Chancen}

  \begin{itemize*}
    \item Hohe Motivation für die Arbeit wird für das Training genutzt.
    \item Körperliche Fitness hilft, beruflichen Stress besser auszuhalten.
    \item Höheres Selbstvertrauen und Prestige durch körperliche Fitness hilft in der Arbeitswelt.
  \end{itemize*}

\subsection{Schnittstelle Arbeit--Pendeln}

\minisec{Herausforderung}

  \begin{itemize*}
    \item Der Dresscode bei der Arbeit entspricht nicht der funktionellen Bekleidung auf dem RR.
    \item Körperhygiene nach schweisstreibendem Pendeln am Arbeitsplatz sichern (Duschen).
    \item Transport von Unterlagen, Lunch und Umziehsachen.
  \end{itemize*}

\minisec{Chance}

  \begin{itemize*}
    \item Pendeln als Grenze zwischen Arbeit und Freizeit (Kontamination der Freizeit durch Berufsstress).
    \item Möglichkeit der geistigen Verarbeitung von Berufsstress während Pendelzeit.
    \item Einsparungen durch Pendeln mit RR (Benzin, Parkplatz, Kosten für ÖV).
    \item Querfinanzierung des Hobbys durch steuerliche Abzugsmöglichkeiten des Fahrrades.
  \end{itemize*}

Es wird in Zukunft so sein, dass wir mehr und mehr auf das herkömmliche
individuelle Pendeln mit einem Auto werden verzichten müssen (oder aber auch:
tatsächlich den wahren Preis dafür zahlen müssen, individuelle und
Umwelt-Kosten miteingerechnet). Jeder, der sich da schon jetzt Alternativen
überlegt handelt vorausschauend und Zukunfsbewusst. Selbst die Hochburgen der
Automobilindustrie wie BMW denken um und unterstützen das Pendeln mit Fahrrad
\cite{Hage2018bmwfahrrad}.

\subsection{Schnittstelle Training--Pendeln}

\minisec{Herausforderung}

  \begin{itemize*}
    \item Periodisierung des Trainings beisst sich mit der Regelmässigkeit des Pendelns.
    \item Auf tageszeitliches Form kann keine Rücksicht genommen werden.
    \item Distanz zwischen Arbeits- und Wohnortes entspricht nicht dem angestrebten Trainingsvolumen.
  \end{itemize*}

\minisec{Chance}

  \begin{itemize*}
    \item Durch die gezwungene Regelmässigkeit des Pendelns kommen erhebliche Kilometerleistungen zusammen.
    \item Durch Simultanität kann Zeit gespart werden.
    \item Während des Pendelns können Trainingseinzeiten eingebaut werden.
    \item Durch Tageszeit, Witterung und suboptimale Bedingungen wird Rennhärte trainiert.
  \end{itemize*}

Das Buch folgt dabei nicht der obigen -- eher künstlichen -- Struktur, sondern versucht obige Punkte möglichst sinnvoll zu bündeln.


% Stimmen von Betroffenen:
% 
% \shorthandoff{"}
% "Mir geht es oft so, dass ich, bevor ich 'aufsitze', denke,
% '...wie bescheuert musst Du eigentlich sein, bei der Dunkelheit und Kälte...',
% aber wenn ich dann 5 Minuten gefahren bin, oder auf der Arbeit angekommen bin,
% weiß ich warum ich geradelt bin."
% \cite{iglimann2011rennradnews}
% 
% "Die 2x täglichen 20 Minuten frische Luft sind wunderbar!
% Werd ich nass, so what, ich trockne ja auch wieder.
% Gibt keine schönere Genugtuung, als an einem kilometerlangen Rattenschwanz an warmgepupsten, im Stau stehenden PKW entlangzudefilieren"
% \cite{efix2011rennradnews}
% 
% 
% "Ich fühle mich einfach besser mit dem Rad, brauche keinen Parklplatz suchen und bin auf dem Heimweg schneller als mit dem Auto.
% UND!!!!! Wenn ich die Firma verlasse ist der Heimweg schon Hobby/Freizeit,
% und ich muß nicht ERST mit dem Auto nach Hause."
% \cite{littlechex2011rennradnews}
% \shorthandon{"}



