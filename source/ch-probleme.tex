\chapter{Probleme des realen Lebens}

\section{Weitere Informationen}

Die Fahrstrecke kann genutzt werden, um Techniken (Haltung, Trittfrequenz, Wiegeschritt) zu üben.

Täglich an den gleichen Ort zu fahren, hat den Vorteil, dass man Feintuning betreiben kann.
Es empfiehlt sich, gefährliche Kreuzungen oder Strecken allenfalls zu umfahren oder eine Alternative zu suchen.
Auch kann mit der Abfahrtszeit gespielt werden. Schon 10 Minuten früher oder später kann bewirken, dass der Verkehr deutlich weniger ist oder weniger Lastwagen unterwegs sind.

Schlaglöcher sind mit der Zeit sehr vertraut und man kann schon frühzeitig eine andere Linie einschlagen.


\section{Fahrradtransport mit öffentlichen Verkehrsmitteln}

Die hier zusammengetragenen Informationen beziehen sich auf den Nahverkehr.
Oft sind abweichende Bestimmung im Bezug auf die Region oder des lokalen Verkehrsverbundes vorhanden.
Ein Problem ist auch, dass die einfache Fahrradmitnahme gerade zur Pendlerzeit oft nicht erlaubt oder
von (wenig planbarer) Einschätzung des Zugspersonal abhängig ist.
Insgesamt unproblematischer scheint es zu sein, das Fahrrad teildemontiert in eine entsprechende Tasche zu packen und als Handgepäck mitzuführen.

\subsection{Deutsche Bahn (http://www.bahn.de)}
Im Nahverkehr mit Fahrradsymbol kann das Fahrrad mitgenommen werden.
Zeiten mit Berufsverkehr sollen jedoch gemieden werden.
Eine ausdrückliches Mitnahmerecht besteht nicht und das letzte Wort hat das Zugspersonal. Zudem gibt es unterschiedliche Regelungen je nach Region.

Es gibt im DB Nahverkehr eine Fahrradtageskarte für \EUR{5}. Innerhalb von Verkehrsverbünden existieren teilweise abweichende Tarifbestimungen.

Als Handgepäck: Auf der Website der Bahn finden sich keine ausdrücklichen Angaben zur Mitnahme des verpackten, teilzerlegten Fahrrades.
Erfahrungsberichte bestätigen aber die Möglichkeit der Mitnahme.

\subsection{Österreichische Bundesbahn (http://www.oebb.at)}
Im Nahverkehr kann die Fahrradmitnahme nur bei genügend freien Stellplätzen erfolgen.
Ein Fahrradticket kostet 10\% des Vollpreises der 2. Klasse, mindestens aber \EUR{2}.
Es gibt zudem Wochen- und Monatskarten.

Die Mitnahme von teildemontierten und verpackten Fahrrädern ist in allen Zügen kostenfrei möglich.

\subsection{Schweizerische Bundesbahnen (http://www.sbb.ch)}
Eine Fahrradmitnahme ist im Nahverkehr (S-Bahnen) möglich, allerdings von Montag bis Freitag nur von 8 bis 16 Uhr und von 19 bis 6 Uhr.
Ein Fahrrad braucht grundsätzlich das gleiche Ticket wie der Passagier, d.h. den vollen Preis ohne Halbtax-Abo.
Günstiger geht es mit dem Velo-Billet, das es als Velo-Tageskarte (CHF\,18) oder sog. Velo-Pass (1 Jahr, CHF\,220) gibt.

Wenn Sie Ihr Velo in einer Tragetasche verpacken, können Sie es kostenlos als Handgepäck im Zug mitnehmen.
Jede Hülle wird akzeptiert. Das Vorderrad muss demontiert und zusammen mit dem Velo in der Transporthülle verpackt sein.
Verstauen Sie Ihr verpacktes Velo während der Zugfahrt unter bzw. über dem Sitz oder im Einsteigebereich.
Wenn Sie Ihre Velotragetasche auf einem Sitzplatz deponieren möchten, müssen Sie zusätzlich ein halbes Billett lösen.


\section{Fahren trotz körperlichen Symptomen}

\subsection{Zeichen von Uebertraining}

\subsection{Schlafmangel}

\subsection{Fahren mit einer Infektion}

\subsection{Fahren mit Restalkohol}

\subsection{Radfahren mit Kater}
\cite{Hurford2015hangover}

Mögliche Faktoren für Kater-Symptome \cite{swift1998alcohol} sind direkte Effekte des Alkohols.
Dehydratation, Elektrolytengleisung, Gastrointestinale Probleme, tiefer Blutzucker, Schlafstörung
Indirekte Wirkungen des Alkoholkonsumes:
Alkoholenzugssymptome, Abbauprodukte des Alkoholabbaus (Acetaldehyd)
Zusätzlich wirken noch nicht alkoholbezogene Faktoren: Methanol, Fuselalkohole, Nikotin

Flüssigkeit mit Elektrolyten (Konkret?)

Gegen Kopf- und Gleiderschmerzen helfen konventionelle, rezeptfrei erhältliche Schmerzmittel.
Speziell geeignet bei Kater-Kopfschmerzen ist \textsl{Ibuprofen}
(enthalten beispielsweise in Brufen, Contra Schmerz und Dolocyl).
\textsl{Acetylsalicylsäure} (Aspirin) geht auch, kann aber zusätzlich auf den Magen schlagen.
Und wenn wir gerade dabei sind: \textsl{Ibuprofen} wie auch \textsl{Acetylsalicylsäure}
sind Medikamente, die nicht unter die Kategorie Doping fallen \cite{nada2014erlaubtemedikamente}.
Das wären also Medikamente, die in einer solchen Situation vor einem Wettkampf genommen werden könnten.
Abgeraten wird von Paracetamol (auch in vielen Grippemitteln enthalten),
weil \textsl{Paracetamol} den gleichen Abbauweg wie Alkohol nimmt und so zusätzlich die Leber belastet.

Kohlenhydrate (Blutzuckersenkung durch Alkohol)

Niedrige Belastung, hohe Kadenz (Laktat-Abbau)

Vermeidung Intervalle wg. Kopfschmerzen und Bauchbeschwerden

