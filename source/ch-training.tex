\chapter{Pendeln und Training}

\shorthandoff{"}
    \dictum[\citeNP{sollritchey2011rennradnews}]{
        "Ich fahre jeden Tag 33 km einfach zur Arbeit und das bei JEDEM Wetter.
        [\ldots]. Im Winter gibt es kein besseres Training,
        ohne Arbeit würde ich diese km Leistung niemals fahren"
        }
\shorthandon{"}

\section{Problematik}

Bei 100prozentigem Arbeitspensum und allenfalls Familie noch genügend Zeit
für ein seriöses Rennrad-Training aufzubringen ist eine Herausforderung.
Eine mögliche Lösung ist das regelmässige Pendeln mit dem Rennrad (road
bike commuting). In diesem Artikel werden die Schwierigkeiten und mögliche
Lösungen zum Pendeln mit Rennrad (PmRR) dargestellt. Der Autor kann dabei
auf mehrere Jahre Erfahrung mit einem Wochenschnitt von ca. 100 km pendelnd
zurückschauen.

Pendeln zum Arbeitsplatz und Rennrad-Training sind zwei unterschiedliche
Tätigkeiten, die sich nur in einem kleinen Bereich überschneiden, nämlich
in der Eigenschaft, dass man sich (möglichst schnell) vom Punkt A nach Punkt
B bewegt.

In vielen anderen Bereichen sind hier konträre Ziele: beim Pendeln will man
möglichst grosse Flexibilität, gepaart mit möglichst viel Bequemlichkeit.
Man muss oft Dinge (Unterlagen, Bücher, Equipment) transportieren. Der
Arbetisplatz stellt Anforderung an Erscheinungsbild (Hygiene, Kleidung). Der
Arbeitsplatz muss pünktlich erreicht werden -- dies auch bei schlechtem
Wetter oder bei Dunkelheit. Auch will man pünktlich wieder zu Hause sein.

Beim Rennrad-Training will man möglichst \emph{wenig} mitnehmen auf einem
Rad, dass möglichst leicht ist. Die Kleidung soll für das Rannradfahren sehr
funktional sein. Um den Trainingeffekt zu optimieren ist es unvermeidlich,
dass man Schwitzt. Rennradfahren bei Dunkelheit, viel Verkehr oder schlechten
Wetter wird -- wenn möglich -- vermieden.

Vorteile:
\begin{itemize}
        \item Zeit, die man für den Arbeitsweg aufbringt wird als Trainingszeit genützt.
        \item Das Training wird am Tag gesplittet.
        \item Allenfalls Kostenersparnis gegenüber der Benutzung von anderen privaten Verkehrsmitteln (Auto) oder öffentlichem Verkehr.
        \item Keine Abhängigkeit von öffentlichem Verkehr.
\end{itemize}
\section{Faktoren des Trainings}

Häufigkeit, Dauer, Intensität

\section{Häufigkeit}

\section{Dauer}

\begin{table}
        \centering
        \begin{tabular}{ccccccccccc}
                \toprule
&	5	& 10	& 15	& 20	& 25	& 30 & 35	& 40	& 45	& 50\\
    \midrule
1 &	480	& 960	& 1440	& 1920	& 2400	& 2880	& 3360	& 3840	& 4320	& 4800 \\
2 &	960 &	1920 &	2880 &	3840 &	4800 &	5760 &	6720 &	7680 &	8640 &	9600 \\
3 &	1440 & 	2880 &	4320 &	5760 &	7200 &	8640 &	10080 &	11520 &	12960 &	14400 \\
4 &	1920 &	3840 &	5760 &	7680 &	9600 &	11520 &	13440 &	15360 &	17280 &	19200 \\
5 &	2400& 	4800 &	7200 &	9600 &	12000 &	14400 &	16800 &	19200 &	21600 &	24000 \\
\bottomrule
        \end{tabular}
        \caption{Jahreskilometer abhängig von der Strecke Wohnort-Arbeitsort sowie der Anzahl Tage, an denen mit dem Rad gependelt wird.}
        \label{tab:jahreskilometer}
\end{table}

\section{Intensität}



