\chapter{Organisation}

\section{Vorbereitung ist alles}

Vorbereitung ist der Schlüssel für Effizienz.
Die Vorbereitung für ein Pendeln mit dem Rennvelo

Ein Tagesablauf mit optimaler Verbindung von Pendeln, Training und Arbeit.

\subsection{Vorabend}


\section{Wie lange ist die Fahrzeit?}

Die Fahrzeit berechnet sich $t = s/v$.
Für einen Arbeitsweg von 10\,km und einer Durchschnittsgeschwindigkeit von 30\,km/h kommt man also auf $10/30=0.33$.
Multipliziert mit 60 ergibt die Zeit in Minuten, hier also 20\,min, siehe Tabelle \ref{tab:fahrzeit}.

Die Durchschnittsgeschwindigkeit auf dem Rennvelo ist stark abhängig von der Strecke (flach vs. hüglig) und von den Windverhältnissen.
Kräftiger Gegenwind kann die Durchschnittsgeschwindigkeit um 5\,km/h drücken.
Hier hilft die vorgängie Konsultation des Wetterberichtes.
Muss man aufgrund der Witterungsverhältnissen einmal auf's MTB umsteigen, dann ist man ebenfalls 5 -- 10\,km/h langsamer.
Die Tabelle \ref{tab:fahrzeit} hilft, sich dabei zu orientieren.
So kann abgeschätzt werden, wieviel länger man für den Arbeitsweg braucht,
sollten einem einmal die Witterung auf das MTB zwingen.

\begin{table}
        \centering
        \begin{tabular}{cccccc}
                \toprule
            &	10\,km	&   20\,km	& 30\,km	&   40\,km	& 50\,km    \\
    \midrule
20\,km/h	&   30      &	60	    & 90        &   120	    & 150       \\
25\,km/h	&   24      &	48 &	72 &	96 &	120  \\
30\,km/h	&   20      &	40 & 	60& 	80 &	100 \\
35\,km/h	&   17      &	34 &	51& 	69 &	86 \\
40\,km/h	&   15      &	30 &	45& 	60 &	75 \\
\bottomrule
        \end{tabular}
        \caption{Fahrzeit in Minuten, abhängig von der Distanz und der Durchschnittsgeschwindigkeit.
        (MTB minus 5 -- 10\,km/h, Gegenwind minus 5\,km/h).
        Anwendung:
        wird die Strecke zur Arbeit 20\,km mit den Rennvelo bei einer Durchschnittgeschwindigkeit von
        30\,km/h gefahren,
        dann wird man mit dem MTB wohl 8 Minuten (48 - 40) länger haben.}
        \label{tab:fahrzeit}
\end{table}

\section{Transportierendes Material}

\begin{table}
  \centering
  \begin{tabular}{ll}
    \toprule
        Was?    & Kommentar \\
    \midrule
        Schlüssel, Handy, Geldbörse     & Essentials \\
        Notfallwerzeug für Panne        & Siehe näheres unter entsprechendem Abschnitt \\
        Regenschutz, Überschuhe         & Fahrradbezogene Dinge \\
        Wäsche, frisches Hemd           & jeden Tag, nach dem Duschen \\ 
        Verpflegung Tag                 & Entfällt bei Kantinenverpflegung \\
        Geschäftsunterlagen             & Berufsbezogene Unterlagen \\
    \bottomrule
  \end{tabular}
  \caption{Eine Aufstellung der Dinge, die täglich transportiert werden müssen.}
  \label{tab:transportmaterial}
\end{table}


