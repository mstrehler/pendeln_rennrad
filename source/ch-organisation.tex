\chapter{Organisation}

\section{Wie lange ist die Fahrzeit?}

Die Fahrzeit berechnet sich $t = s/v$. Für einen Arbeitsweg von 10\,km und einer Durchschnittsgeschwindigkeit von 30\,km/h kommt man also auf $10/30=0.33$.
Multipliziert mit 60 ergibt die Zeit in Minuten, hier also 20\,min, siehe Tabelle \ref{tab:fahrzeit}.
Die Durchschnittsgeschwindigkeit auf dem Rennvelo ist stark abhängig von der Strecke (flach vs. hüglig) und von den Windverhältnissen.
Kräftiger Gegenwind kann die Durchschnittsgeschwindigkeit gut um 5\,km/h drücken.
Muss man aufgrund der Witterungsverhältnissen einmal auf's MTB umsteigen, dann ist man ebenfalls 5 -- 10\,km/h langsamer.
Die Tabelle \ref{tab:fahrzeit} hilft, sich dabei zu orientieren.

\begin{table}
        \centering
        \begin{tabular}{cccccc}
                \toprule
            &	10\,km	&   20\,km	& 30\,km	&   40\,km	& 50\,km    \\
    \midrule
20\,km/h	&   30      &	60	    & 90        &   120	    & 150       \\
25\,km/h	&   24      &	48 &	72 &	96 &	120  \\
30\,km/h	&   20      &	40 & 	60& 	80 &	100 \\
35\,km/h	&   17      &	34 &	51& 	69 &	86 \\
40\,km/h	&   15      &	30 &	45& 	60 &	75 \\
\bottomrule
        \end{tabular}
        \caption{Fahrzeit in Minuten, abhängig von der Distanz und der Durchschnittsgeschwindigkeit.
        Mit einem Mountainbike ist man zwischen 5 -- 10\,km/h langsamer.
        So kann auch abgeschätzt werden, wieviel länger man für den Arbeitsweg braucht,
        sollten einem einmal die Witterung auf das MTB zwingen.
        Ebenfals rund 5\,km/h langsamer ist man bei starkem Gegenwind (Wetterbericht!).}
        \label{tab:fahrzeit}
\end{table}

\section{Transportierendes Material}

