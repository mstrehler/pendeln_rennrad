\chapter{Organisation}

\section{Vorbereitung ist alles}

Vorbereitung ist der Schlüssel für Effizienz.
Die Vorbereitung für ein Pendeln mit dem Rennvelo

Ein Tagesablauf mit optimaler Verbindung von Pendeln, Training und Arbeit.

\subsection{Vorabend}

Das Ziel ist, dass jeder Handgriff sitzt.
Eine perfekt sitzende Routine spart viel Zeit, Fehler und Motivation.
Etwas was so geläufig ist wie Zähneputzen muss nicht mehr im Hirn verarbeitet werden und das spart Energie.
Durch das, dass sich weniger Fehler einschleichen, bleibt der Frust klein und damit die Motivation hoch.
Durch die Routine kommt man auch automatisch in den richtigen Modus und dysfunktionale Kognitionen wie
<<Es ist aber morgen ziemlich kalt!>>, <<Das wird aber ein anstrengender Tag, vielleicht gehe ich doch mit dem Zug>> kommen gar nicht erst hoch.

Man soll sich eine Reihenfolge zurechtlegen und die penibel einhalten.
Bei der Vorbereitung am Vorabend muss man dabei sich von der Peripherie zum Zentrum vorarbeiten.

\begin{enumerate}
  \item Informationen sammeln:
        Wie wird das Wetter?
        Habe ich relevanten Rücken- oder Gegenwind?
        Wie sind die Strassenverhältnisse?
        Was brauche ich morgen zur Arbeit?
        Wann muss ich dort sein?
    Diese Daten werden für die folgenden Vorbereitungen gebraucht.
  \item Präparation des Rennvelos: Reifendruck, Bidon, Licht.
    Hier können allenfalls die gesammelten Informationen einfliessen. Muss ich ev. das Rad wechseln und das MTB nehmen?
  \item Ausrüstung: Helm, Handschuhe, Navi. Sind die elektronischen Geräte geladen oder am Kabel?
  \item Kleidung: Schuhe, Hosen, Trikot -- entsprechend der Witterung (s.o.) angepasst.
    Ev. habe ich beim MTB andere Schuhe -- deshalb lohnt sich hier auch eine fixe Reihenfolge der Tätigkeit.
  \item Gepäck im Rucksack: Ersatzwäsche usw.
\end{enumerate}

\subsection{Am Morgen}

\subsection{Gleich nach der Ankunft nach Hause}

\subsection{Am Wochenende}




\section{Wie lange ist die Fahrzeit?}

Die Fahrzeit berechnet sich $t = s/v$.
Für einen Arbeitsweg von 10\,km und einer Durchschnittsgeschwindigkeit von 30\,km/h kommt man also auf $10/30=0.33$.
Multipliziert mit 60 ergibt die Zeit in Minuten, hier also 20\,min, siehe Tabelle \ref{tab:fahrzeit}.

Die Durchschnittsgeschwindigkeit auf dem Rennvelo ist stark abhängig von der Strecke (flach vs. hüglig) und von den Windverhältnissen.
Kräftiger Gegenwind kann die Durchschnittsgeschwindigkeit um 5\,km/h drücken.
Hier hilft die vorgängie Konsultation des Wetterberichtes.
Muss man aufgrund der Witterungsverhältnissen einmal auf's MTB umsteigen, dann ist man ebenfalls 5 -- 10\,km/h langsamer.
Die Tabelle \ref{tab:fahrzeit} hilft, sich dabei zu orientieren.
So kann abgeschätzt werden, wieviel länger man für den Arbeitsweg braucht,
sollten einem einmal die Witterung auf das MTB zwingen.

\begin{table}
        \centering
        \begin{tabular}{cccccc}
                \toprule
            &	10\,km	&   20\,km	& 30\,km	&   40\,km	& 50\,km    \\
    \midrule
20\,km/h	&   30      &	60	    & 90        &   120	    & 150       \\
25\,km/h	&   24      &	48 &	72 &	96 &	120  \\
30\,km/h	&   20      &	40 & 	60& 	80 &	100 \\
35\,km/h	&   17      &	34 &	51& 	69 &	86 \\
40\,km/h	&   15      &	30 &	45& 	60 &	75 \\
\bottomrule
        \end{tabular}
        \caption{Fahrzeit in Minuten, abhängig von der Distanz und der Durchschnittsgeschwindigkeit.
        (MTB minus 5 -- 10\,km/h, Gegenwind minus 5\,km/h).
        Anwendung:
        wird die Strecke zur Arbeit 20\,km mit den Rennvelo bei einer Durchschnittgeschwindigkeit von
        30\,km/h gefahren,
        dann wird man mit dem MTB wohl 8 Minuten (48 - 40) länger haben.}
        \label{tab:fahrzeit}
\end{table}

\section{Transportierendes Material}

\begin{table}
  \centering
  \begin{tabular}{ll}
    \toprule
        Was?    & Kommentar \\
    \midrule
        Schlüssel, Handy, Geldbörse     & Essentials \\
        Notfallwerzeug für Panne        & Siehe näheres unter entsprechendem Abschnitt \\
        Regenschutz, Überschuhe         & Fahrradbezogene Dinge \\
        Wäsche, frisches Hemd           & jeden Tag, nach dem Duschen \\ 
        Verpflegung Tag                 & Entfällt bei Kantinenverpflegung \\
        Geschäftsunterlagen             & Berufsbezogene Unterlagen \\
    \bottomrule
  \end{tabular}
  \caption{Eine Aufstellung der Dinge, die täglich transportiert werden müssen.}
  \label{tab:transportmaterial}
\end{table}


